\documentclass{article}
\usepackage{cite, listing, graphicx, subfigure, amsmath}
\title{Sun RPC Programming: A Simple Distributed File System}
\author{Jorge G\'omez}

\begin{document}
\maketitle
\section{Answers to Assignment Questions}
  \begin{itemize}
    \item Are the file operations idempotent?\\
      The functions are all idempotent because no matter how many times they are
      called they will return the same result.
    \item List at least three differences between local and remote procedure calls.
	\begin{itemize}
          \item Remote calls can involve a remote processor with it's own register file
            and cache values.
          \item Remote calls involve putting the arguments and return values of a function
            in a serialized packet form for transfer over a network.
          \item The call to a remote procedure must block execution on the caller.
        \end{itemize}
    \item List the basic steps that take place on the client and server during an RPC.\\
      The client will call a local procedure that initiates a network transfer to the remote
      host. The host then process the request and returns a value to the network. The local
      procedure then returns the value to the caller.
    \item Does memory need to be freed on the client of server?\\
      Memory should be freed on the server so that multiple calls to the stub do not have
      erroneous data contained within. Also, the server can run constantly and good memory
      management is important to avoid memory leaks.
  \end{itemize}
\end{document}
